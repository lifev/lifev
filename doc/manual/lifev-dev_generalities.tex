
\chapter{Generalities}
\label{cha:generalities}

\section{Scope of the document}
\label{sec:scope-document}

This is a ``working document'' meant to drive the development of the
software library \thelibrary. It is an informal document. All the
people involved in the library development (from now on called the
\emph{authors}) are supposed to freely contribute to its writing.

The major objectives of this document is manifold:
\begin{enumerate}
\item Set up a set of rules regarding program development, coding rules
  and standards, which must be followed by the software developers;
\item Provide the architectural design for the libraries and classes
  to be developed, with a special stress on public interfaces;
\item Provide a subtask subdivision for the work to be done and a
  (possibly accurate) \emph{scheduling}.
\end{enumerate}

\section{Language and nomenclature convection} 
\label{sec:lang-nomencl-conv}

Because of many practical reasons, the document will be written in
English, or at least the best English the authors can manage.

We will use \texttt{typesetting font style} to indicate parts of
actual computer code or name of variables, types, etc..
\textbf{Boldface} is used to mark portion containing rules which
should be followed during program development and \emph{emphasised}
text to indicate important concepts and nomenclature.
 
\section{Software Management} 
\label{sec:software-management}

Being this software the result of the work of many people working in a
(hopefully) coordinated fashion, some rules for software management
must be set and agreed upon.

A related problem is how to allow for the most ample discussion
between the authors and, at the same time, coordinate  software
production.

The idea is that, while software development will be subdivided into
subtask, whose responsibility will be assigned to a specific person,
or group of persons working together, all major decisions should be
collectively discussed. Since the authors operates far away from each
other, we need to held regular meetings and use extensively Web means
such as e-mails. Luca Formaggia may also make stop-overs at Verona or
Milan on Mondays, when stronger coordination with the Italian part of
the team is needed.
   

\subsection{Some conventions}
\label{sec:some-conventions}

The software source, its documentation and all related documents (this
one included) will be kept in a repository under revision control
using CVS\footnote{CVS stands for Concurrent Version System}.  

CVS keywords like \verb!Id! and \verb!Log! should not be included in source files,
they cause many unnecessary conflicts at update/commit time. Use 
\verb!cvs log! to get the information given by \verb!Id! or \verb!Log!.
